\documentclass{article}

% math, graphics, and formatting
\usepackage{amsmath,amsbsy,amssymb,amsthm,fullpage,graphicx,pgfplots,breqn}
\usepackage{verbatim,mathtools}

% physics
\usepackage{physics}
\newcommand{\h}{\hbar}
\renewcommand{\vec}{\mathbf}

% Times New Roman
\usepackage{newtxtext}
\usepackage{newtxmath}
\renewcommand{\mathbb}{\varmathbb}
%\usepackage{libertine}
%\usepackage[libertine]{newtxmath}

% section numbering: e.g. 2b (ii)
%\renewcommand\thesection{\arabic{section}}
%\renewcommand\thesubsection{\thesection\alph{subsection}}
%\renewcommand\thesubsubsection{\thesubsection\,(\roman{subsubsection})}

% theorems & proofs
\theoremstyle{definition}
\newtheorem*{prop}{Claim}
\newtheorem*{claim}{Claim}
\newtheorem*{prob}{Problem}
\newtheorem*{obs}{Observation}
\newtheorem*{lemma}{Lemma}
\newtheorem*{thm}{Theorem}
\newtheorem*{disc}{Discussion}
\newtheorem*{defn}{Definition}
\newtheorem*{eg}{Example}

% derivatives
\newcommand{\pp}{\partial}
\newcommand{\pd}[2]{\ensuremath\frac{\pp #1}{\pp #2}}
\newcommand{\pdd}[2]{\ensuremath\frac{\pp^2 #1}{\pp #2^2}}
\renewcommand{\dd}[2]{\ensuremath\frac{d#1}{d#2}}
\newcommand{\ddd}[2]{\ensuremath\frac{d^2 #1}{d #2^2}}
\newcommand{\del}{\nabla}

% sets and complex numbers
\newcommand{\R}{\mathbb R}
\renewcommand{\Re}{\textrm{Re}}
\renewcommand{\Im}{\textrm{Im}}
%\renewcommand{\bar}{\overline}
\newcommand{\nin}{\not\in}
\newcommand{\str}[1]{\ensuremath{\langle #1 \rangle}}

% probability and stats
\renewcommand{\var}{\ensuremath\mathbf{Var}}
\newcommand{\E}{\ensuremath\mathbf{E}}
\renewcommand{\P}{\ensuremath\mathbf{P}}

% measurement
\newcommand{\un}[1]{\;\mathrm{#1}}
\newcommand{\ch}[1]{\mathrm{#1}}

% Greek letters are hard
\newcommand{\ep}{\varepsilon}
\newcommand{\om}{\omega}

% formatting
\renewcommand{\sp}[1]{\;\;\;\text{ #1 }\;\;\;}
\newcommand{\cc}{\texttt}
\newcommand{\plop}[2]{
    \begin{figure}\centering
        \includegraphics[width=0.8\textwidth]{#1}
        \caption{\label{#1}#2}
    \end{figure}
}
\newcommand{\sbsplop}[4]{
    \begin{figure}\centering
        \includegraphics[width=0.49\textwidth]{#1}
        \includegraphics[width=0.49\textwidth]{#2}
        \caption{\label{#1}#3}
        {#4}
    \end{figure}
}

\usepackage{fancyhdr}
\usepackage[margin=1in, headheight=50pt]{geometry}
\pagestyle{fancy}
\lhead{\textbf{Ph21 Set 2}}
\chead{}
\rhead{Aritra Biswas}
\setlength{\headsep}{20pt}

\begin{document}

\section{Theory}

\subsection{Self-consistency of the Fourier series}

\begin{obs}
The following is a valid representation of the Dirac delta function:
\begin{align*}
\sum_{k = -\infty}^\infty e^{iky} = 2\pi \delta(y).
\end{align*}
\end{obs}

\begin{lemma}
$\delta(ay) = \delta(y)/a$.
\end{lemma}

\begin{proof} We start with a change of variables $y \mapsto y/a$,
which leaves the limits of integraiton unchanged:
\begin{align*}
\int_{-\infty}^\infty f(y) \delta(ay) \, dy
&= 
\int_{-\infty}^\infty f(y/a) \delta(y) \, d(y/a)
=
{1 \over a} \int_{-\infty}^\infty f(y/a) \delta(y) \, dy
= {f(0) \over a}
=
\int_{-\infty}^\infty f(y) \left[\delta(y) \over a \right] \, dy.
\end{align*}
\end{proof}

\begin{claim}
The Fourier series, as defined in equations (3) and (2) in the notes,
is self-consistent.
\end{claim}

\begin{proof}
Using the definition of $\tilde h_k$ and $f_k \equiv k/L$,
\begin{align*}
h(x) &= \sum_{k = -\infty}^\infty
\left[
    {1 \over L} \int_0^L h(x') e^{2\pi i k x' / L} \, dx'
\right]
e^{-2\pi i k x / L}
\\&=
\int_0^L h(x') \left[
    {1 \over L} \sum_{k = -\infty}^\infty e^{2\pi i k (x' - x) / L}
\right] \, dx'
\\&=
\int_0^L h(x') \delta(x' - x) \, dx'
= h(x).
\end{align*}
\end{proof}

\subsection{Linear combination of exponentials}

\begin{lemma}
$\sin(\theta + \phi) = \sin\theta\cos\phi + \cos\theta\sin\phi.$
\end{lemma}
\begin{proof}
Using Euler's formula:
\begin{align*}
e^{i(\theta + \phi)} &= e^{i\theta} e^{i\phi} \\
\cos(\theta + \phi) + i \sin(\theta + \phi)
&= (\cos\theta + i \sin\theta)(\cos\phi + i \sin\phi)
\\&=
\cos\theta\cos\phi - \sin\theta\sin\phi
+ i(\sin\theta\cos\phi + \cos\theta\sin\phi).
\end{align*}
Equating imaginary parts yields the desired result.
\end{proof}

\begin{claim}
$A \sin(2\pi x / L + \phi)$ is a linear combination of
$e^{-2\pi i x / L}$ and
$e^{2\pi i x / L}$ over the scalar field $\mathbb C$.
\end{claim}
\begin{proof}
We use the trigonometric identity
above and the definitions of $\sin$ and $\cos$
in terms of complex exponentials (from Euler's formula):
\begin{align*}
A \sin(2\pi x / L + \phi) &= (A \cos\phi) \sin(2\pi x / L)
+ (A\sin\phi) \cos(2\pi x / L)
\\&=
(A\cos\phi) \left(e^{2\pi i x / L} - e^{-2\pi i x / L} \over 2i\right)
+ (A\sin\phi) \left(e^{2\pi i x / L} + e^{-2\pi i x / L} \over 2\right)
\\&=
\left[A (\sin\phi - i \cos\phi) \over 2\right] e^{2\pi i x / L} 
+ \left[A (\sin\phi + i \cos\phi) \over 2\right] e^{-2\pi i x / L}. 
\end{align*}
\end{proof}

\subsection{Redundancy in Fourier coefficients of real functions}

\begin{claim}
For $h(x) \in \mathbb R$, the Fourier coefficients $\tilde h_k$ satisfy
$\tilde h_{-k} = \tilde h^*_k$.
\end{claim}

\begin{proof}
Conjugation is linear: for $A, B \in \mathbb R$ and
$\alpha, \beta \in \mathbb C$, $(A\alpha + B\beta)^* = A\alpha^*
+ B\beta^*$. Therefore, to show that two integral expressions are
conjugates, it suffices to show that their integrands are conjugates:
\begin{align*}
\tilde h_{-k} &= {1 \over L} \int_0^L h(x) e^{-2\pi ikx/L} \, dx
\\&= {1 \over L} \int_0^L h(x) \left(e^{2\pi ikx/L}\right)^* \, dx
\\&= {1 \over L} \int_0^L \left[h(x) e^{2\pi ikx/L}\right]^* \, dx
\\&= \left[ {1 \over L} \int_0^L h(x) e^{2\pi ikx/L} \, dx \right]^*
= \tilde h_k^*.
\end{align*}
\end{proof}

\subsection{Convolution theorem}

\begin{claim}
The Fourier coefficients of the product $H(x) = h^{(1)}(x) h^{(2)}(x)$
are given by the convolution:
\begin{align*}
\tilde H_k = \sum_{k' = -\infty}^\infty \tilde h^{(1)}_{k-k'}
\tilde h^{(2)}_{k'}.
\end{align*}
\end{claim}
\begin{proof}
We express $h^{(1)}(x)$
and $h^{(2)}(x)$ as Fourier series and find the Cauchy product $H(x)$:
\begin{align*}
H(x) &= 
\left[
\sum_{k = -\infty}^\infty \tilde h^{(1)}_{k} e^{-2\pi ikx/L}
\right]
\left[
\sum_{k' = -\infty}^\infty \tilde h^{(2)}_{k'} e^{-2\pi ik'x/L}
\right]
\\&=
\sum_{k = -\infty}^\infty \sum_{k' = -\infty}^\infty
\left[ \tilde h^{(2)}_{k'} e^{-2\pi ik' x/L} \right]
\left[ \tilde h^{(1)}_{k - k'} e^{-2\pi i(k - k') x/L} \right]
\\&=
\sum_{k = -\infty}^\infty \sum_{k' = -\infty}^\infty \tilde
h^{(1)}_{k - k'}
\tilde h^{(2)}_{k'} e^{-2\pi i k x/L}
\\&=
\sum_{k = -\infty}^\infty \tilde H_k e^{-2\pi i k x/L},
\end{align*}
Equating the cofficients of the sum over $k$, we have the desired
result.
\end{proof}

Graphically, the convolution of $\tilde h^{(1)}$ and $\tilde h^{(2)}$
can be interpreted as follows: pick $\tilde h^{(1)}_{k'}$ without loss of
generality, since convolution is commutative. Flip the other function
on the $k'$ axis and add an offset: $\tilde h^{(2)}_{k - k'}$. As we
integrate/sum over $k'$, these two functions are moved together on the same
axis -- as if one cross through the other -- 
and the convolution is the integral/sum of their product at each point.

\subsection{Testing the \texttt{numpy} FFT}

We compare analytical and numerical methods of obtaining the Fourier
series of the following two functions:
\begin{align*}
g(t) = A \cos (ft + \varphi) + C \sp{and}
h(t) = A \exp [-B (t - L/2)^2 ].
\end{align*}

\subsubsection{Cosine}

Since $\cos t$ has period $2\pi$, $g(t)$ has period $2\pi/f \equiv L$.
The Fourier coefficients are:
\begin{align*}
\tilde g_k
&= {1 \over L} \int_0^L [A \cos(ft + \varphi) + C]
e^{2\pi i k t / L} \, dt \\
&= {f \over 2\pi} \int_0^{2\pi/f} [A \cos(ft + \varphi) + C]
e^{i k f t} \, dt \\
&= {f \over 2\pi}
\left[
A \int_0^{2\pi/f} \cos(ft + \varphi)
e^{i k f t} \, dt
+ C  \int_0^{2\pi/f}
e^{i k f t} \, dt
\right]
=
\begin{cases}
C & k = 0 \\
{A \over 2} e^{-i\varphi} & k = \pm 1 \\
0 & k \not= 0, \pm 1.
\end{cases}
\end{align*}
The first integrand is a product of two waves -- a cosine with
frequency $f$ and a complex exponential with frequency $kf$ --
over full periods of both waves.
They will destructively interfere everywhere except when $k = \pm 1$,
creating matching frequencies.
The second integral evaluates to $f/2\pi$ at $k = 0$, since the integrand
is unity, and to 0 everywhere else.
We expect isolated peaks at $k = \pm 1$ and a larger
peak at $k = 0$. The FFT implementation in
\cc{numpy} makes it easy to convert the $k$-axis to a
frequency axis (cycles per second, not angular), whereas $f$ as given
is an angular frequency, so we should see the
isolated peaks at $\bar f = 0, \pm f/2\pi$.

A key difference between our analytical solution and the numerical
implementation is that here, we chose $L$ to be the period of the cosine.
In \cc{numpy}, $L$ will be the length of our dataset. As long as $L$
is large enough to capture a full oscillation, we should see the
same peaks in the transform, but their height will scale with $L$ as
the integration interval will be larger.

Figure \ref{g.pdf} shows $g(t)$ and figure \ref{Fg.pdf} shows its
transform.
As expected, there are two isolated peaks in the Fourier transform
$\tilde g(\bar f)$ at $\bar f = \pm f/2\pi \approx \pm 0.32$ and
a tall peak at $\bar f = 0$. The width
of the peaks at $\bar f = \pm f/2\pi$
is due to the sampling -- the input data from
$g(t)$ does not perfectly trace a cosine. Inverse-transforming the data
in figure \ref{Fg.pdf} returns figure \ref{g.pdf} as expected.
\plop{g.pdf}{$g(t)$ vs. $t$ with paramaters
$A = 2$, $f = 2$, $\varphi = \pi/2$, and $C = 0$.}
\plop{Fg.pdf}{The Fourier transform $\tilde g(\bar f)$ of the cosine
vs. $\bar f$.}


\subsubsection{Gaussian}

We are given that $h(t)$ is $L$-periodic for small enough $B$ -- that is,
for small enough standard deviation, we can repeat Gaussians
centered at multiples of the mean without the distributions bleeding into
each other significantly. The Fourier coefficients of the Gaussian
are given by:
\begin{align*}
\tilde h_k &= {1 \over L} \int_0^L A e^{-B (t - L/2)^2}
e^{2\pi i k t / L} \, dt \\
&=
{A\sqrt\pi \over 2L \sqrt B}
\left(
    \erf \gamma_+ + \erf \gamma_-
\right)
e^{i\pi k}
\exp\left(
    -{\pi^2 k^2 \over B L^2}
\right)
\sp{where}
\gamma_{\pm} \equiv {BL^2 \pm 2\pi i k \over 2L \sqrt B},
\end{align*}
so we expect oscillations due to the $e^{i\pi k}$ factor
in a Gaussian envelope centered at $k = 0$.
Figure \ref{h.pdf} shows the Gaussian, and figure \ref{Fh.pdf} shows
its Fourier transform. The shapes are as expected, and applying the transform
and inverse transform in succession returns figure \ref{h.pdf}.
\plop{h.pdf}{$h(t)$ vs. $t$ with parameters $L = 50$ and $B = 0.25$.}
\plop{Fh.pdf}{The Fourier transform $\tilde h(\bar f)$ of the Gaussian
vs. $\bar f$.}

\section{Uniformly-sampled Arecibo data}

\subsection{Isolating the signal frequency}

The Arecibo dataset is sampled at 1 ms intervals, so \cc{numpy.fft.fftfreq}
will return an $f$-axis in cycles per millisecond, or thousands
of cycles per second (kHz). We multiply
by $10^{-3}$ to get our signal in MHz.

Figure \ref{Fa1.pdf} shows the transform of the Arecibo data. A single peak,
isolated from the noise, is visible. Figure \ref{Fa1_scaled.pdf} shows
a closer view of this peak. At this scale, the finite width of the peak
is visible. The peak is maximized at $|f - 1420\un{MHz}| =
1.37 \times 10^{-4}\un{MHz}$,
so we detect a signal at $1420 \pm 1.37 \times 10^{-4}
\un{MHz}$. Here, the $\pm$ does not denote an uncertainty, but is a result
of the symmetry of the real-valued Fourier transform.
\plop{Fa1.pdf}{Fourier transform of the Arecibo data. Frequencies [MHz]
on the $x$-scale have been shifted such that zero corresponds to
$1420\un{MHz}$.}
\plop{Fa1_scaled.pdf}{Closer view of the single peak visible in figure
\ref{Fa1.pdf}.}

\subsection{Gaussian envelope}

If the signal is a perfect sinusoid multiplied by a Gaussian envelope,
then we expect the Fourier transform of the signal to be the convolution of
a delta function (transform of the sinusoid) and a Gaussian-enveloped
wave (transform of the Gaussian as in figure \ref{Fh.pdf}).

Consider the graphical interpretation of convolution as explained in section
1.4. If we hold the Gaussian transform and move a delta function along the
same axis, we note that the product of the two functions will be nonzero
only as the delta function (which is nonzero at an isolated location)
is moving through the Gaussian transform. The resulting
convolution will therefore have the same width as the Gaussian transform.

Our Gaussian is of the form:
\begin{align*}
h(t) = \exp\left[-{(t - t_0)^2 \over \Delta t^2}\right].
\end{align*}
To determine $\Delta t$, we plot several transform $\tilde h(f)$
generated from different $\Delta t$ and compare the transform width
to the signal. We saw in section 1.5.2 that a Gaussian with mean
$L/2$ yields a transform centered around $k = 0$, so we will
set $t_0 = (32768/2) \un{ms}$ and shift the resulting transform
to $|f - 1420\un{MHz}| = 1.37 \times 10^{-4} \un{MHz}$, the central
frequency of the signal.

Figure \ref{Fa1_comp.pdf} shows this comparison for $\Delta t =
{t_0 \over 2}, {t_0 \over 4}, {t_0 \over 6}, {t_0 \over 8}$. Of these,
the Gaussian transform for $\Delta t = {t_0 \over 4} = 4096\un{ms}$
matches the signal transform width best.
\plop{Fa1_comp.pdf}{Comparison of the signal transform, $\tilde g(f)$,
with the transforms of Gaussian envelopes, $\tilde h(f)$, with various
widths $\Delta t$.}

\section{Lomb-Scargle routine for unequally sampled data}

\end{document}
